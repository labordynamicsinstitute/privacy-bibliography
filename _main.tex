\documentclass[]{article}
\usepackage{lmodern}
\usepackage{amssymb,amsmath}
\usepackage{ifxetex,ifluatex}
\usepackage{fixltx2e} % provides \textsubscript
\ifnum 0\ifxetex 1\fi\ifluatex 1\fi=0 % if pdftex
  \usepackage[T1]{fontenc}
  \usepackage[utf8]{inputenc}
\else % if luatex or xelatex
  \ifxetex
    \usepackage{mathspec}
  \else
    \usepackage{fontspec}
  \fi
  \defaultfontfeatures{Ligatures=TeX,Scale=MatchLowercase}
\fi
% use upquote if available, for straight quotes in verbatim environments
\IfFileExists{upquote.sty}{\usepackage{upquote}}{}
% use microtype if available
\IfFileExists{microtype.sty}{%
\usepackage{microtype}
\UseMicrotypeSet[protrusion]{basicmath} % disable protrusion for tt fonts
}{}
\usepackage[margin=1in]{geometry}
\usepackage{hyperref}
\hypersetup{unicode=true,
            pdftitle={Introductory Readings in Formal Privacy for Economists},
            pdfauthor={John M. Abowd, Ian M. Schmutte, William Sexton, and Lars Vilhuber},
            pdfborder={0 0 0},
            breaklinks=true}
\urlstyle{same}  % don't use monospace font for urls
\usepackage{longtable,booktabs}
\usepackage{graphicx,grffile}
\makeatletter
\def\maxwidth{\ifdim\Gin@nat@width>\linewidth\linewidth\else\Gin@nat@width\fi}
\def\maxheight{\ifdim\Gin@nat@height>\textheight\textheight\else\Gin@nat@height\fi}
\makeatother
% Scale images if necessary, so that they will not overflow the page
% margins by default, and it is still possible to overwrite the defaults
% using explicit options in \includegraphics[width, height, ...]{}
\setkeys{Gin}{width=\maxwidth,height=\maxheight,keepaspectratio}
\IfFileExists{parskip.sty}{%
\usepackage{parskip}
}{% else
\setlength{\parindent}{0pt}
\setlength{\parskip}{6pt plus 2pt minus 1pt}
}
\setlength{\emergencystretch}{3em}  % prevent overfull lines
\providecommand{\tightlist}{%
  \setlength{\itemsep}{0pt}\setlength{\parskip}{0pt}}
\setcounter{secnumdepth}{5}
% Redefines (sub)paragraphs to behave more like sections
\ifx\paragraph\undefined\else
\let\oldparagraph\paragraph
\renewcommand{\paragraph}[1]{\oldparagraph{#1}\mbox{}}
\fi
\ifx\subparagraph\undefined\else
\let\oldsubparagraph\subparagraph
\renewcommand{\subparagraph}[1]{\oldsubparagraph{#1}\mbox{}}
\fi

%%% Use protect on footnotes to avoid problems with footnotes in titles
\let\rmarkdownfootnote\footnote%
\def\footnote{\protect\rmarkdownfootnote}

%%% Change title format to be more compact
\usepackage{titling}

% Create subtitle command for use in maketitle
\providecommand{\subtitle}[1]{
  \posttitle{
    \begin{center}\large#1\end{center}
    }
}

\setlength{\droptitle}{-2em}

  \title{Introductory Readings in Formal Privacy for Economists}
    \pretitle{\vspace{\droptitle}\centering\huge}
  \posttitle{\par}
    \author{John M. Abowd, Ian M. Schmutte, William Sexton, and Lars Vilhuber}
    \preauthor{\centering\large\emph}
  \postauthor{\par}
      \predate{\centering\large\emph}
  \postdate{\par}
    \date{2019-05-02}

% addtional packages for the book
\usepackage{chapterbib}

\begin{document}
\maketitle

{
\setcounter{tocdepth}{1}
\tableofcontents
}
\section{Overview}\label{overview}

The purpose of this document is to provide scholars with a comprehensive
list of readings relevant to the economic analysis of formal privacy,
and particularly its application to public statistics. Statistical
agencies and tech giants are rapidly adopting formal privacy models
which make the tradeoff between privacy and data quality precise. The
question then becomes, how much privacy loss should they allow? J. M.
Abowd \& Schmutte
(\protect\hyperlink{ref-AbowdSchmutte:Privacy:AER}{2019}) argue that
this choice ultimately depends on how decision makers weigh the costs of
privacy loss against the benefits of higher-quality data. Making
progress on these questions requires familiarity with new tools from
computer science and statistics, the objectives and policy environment
within which statistical agencies operate, along with the economic
analysis of information.

We have organized these references into a reading course focused on
10-15 primary references in each of six different topics:

\begin{itemize}
\tightlist
\item
  \protect\hyperlink{formal-privacy}{Formal Privacy}
\item
  \protect\hyperlink{policy-and-official-statistics}{Policy and Official
  Statistics}
\item
  \protect\hyperlink{statistical-disclosure-limitation}{Statistical
  Disclosure Limitation}
\item
  \protect\hyperlink{economics-of-privacy}{Economics of Privacy}
\item
  \protect\hyperlink{value-of-privacy-and-data-accuracy}{Value of
  Privacy and Data Accuracy}
\end{itemize}

In the remainder of this document, for each topic, we provide a very
brief description of the papers in the reading course and why we
selected them. In each case, we orient the reader to the key issues,
concepts, and tools in each topic. In addition to this reading course,
we have also curated a much more extensive list of references,
\href{sloan-privacy-bibliography.pdf}{available here}.

\subsection{Contributing}\label{contributing}

We encourage interested readers and researchers to use these readings
for their classes and training, modifying it as needed. You can fork the
source code at
\url{https://github.com/labordynamicsinstitute/privacy-bibliography}.
This document is licensed under the
\href{https://creativecommons.org/licenses/by-nc/4.0/}{Creative Commons
Attribution-NonCommercial 4.0 International Public License (CC BY-NC
4.0)}. Should that license not meet your needs, contact us to discuss
possible modifications or exemption. Please send us any improvements,
corrections, or other contributions, either by e-mail, or via the
``\href{https://github.com/labordynamicsinstitute/privacy-bibliography/issues}{Issues}''
feature on Github.

\subsection{Support}\label{support}

We gratefully acknowledge the support of Alfred
P.~\href{https://sloan.org/grant-detail/6845}{Sloan Foundation Grant
G-2015-13903} and
\href{http://www.nsf.gov/awardsearch/showAward.do?AwardNumber=1131848}{NSF
Grant SES-1131848}.

\subsection{Disclaimer}\label{disclaimer}

The views expressed in this paper are those of the authors and not those
of the U.S. Census Bureau or other sponsors.

\section{Background}\label{background}

We start by providing a short list of background references that frame a
particular set of research topics. J. M. Abowd \& Schmutte
(\protect\hyperlink{ref-AbowdSchmutte:Privacy:AER}{2019}) show how to
combine formal privacy models with the classic theory of public goods to
understand and guide decisions about privacy protection and data
dissemination. For the neophyte, Wood et al.
(\protect\hyperlink{ref-WoodVanderbiltJ.Entertain.Technol.Law2018}{2018})
provide a non-technical introduction to formal privacy models in
general, and differential privacy in particular. Heffetz \& Ligett
(\protect\hyperlink{ref-Heffetz2014}{2014}) also introduce differential
privacy, but targeted toward economists. After reading these
introductory treatments, you should review the textbook treatment of
differential privacy in Dwork \& Roth
(\protect\hyperlink{ref-Dwork:Roth:journal:version:2014}{2014}) ,
focusing on Chapters 1--3. We also recommend consultation of the very
fine tutorial
\href{https://users.cs.duke.edu/~ashwin/index.html\#tutorials}{``Differential
Privacy in the Wild: A tutorial on current practices and open
challenges''} by Michael Hay, Xi He, and Ashwin Machanavajjhala.

To provide a concrete grounding in practical issues of privacy, Jones
(\protect\hyperlink{ref-jones:2017}{2017}) summarizes the history of
privacy breaches and privacy protection in the U.S. statistical system.
It is important to ask how formal privacy models do, and do not, capture
common language and legal interpretations of privacy. As such, we also
recommend a review of some of the laws governing data privacy in the
U.S. , e.g. 13 U.S.~Code (\protect\hyperlink{ref-title13}{1954}) and
H.R.4174 (\protect\hyperlink{ref-cipsea}{2018}) (Confidential
Information Protection and Statistical Efficiency Act, also known as
CIPSEA). A quick review of the Harvard Privacy Tools website (Harvard
University Privacy Tools Project,
\protect\hyperlink{ref-Harvard:PrivacyTools}{2019}) can provide a sense
of how differential privacy is being implemented in various settings.
Goroff (\protect\hyperlink{ref-Goroff2015}{2015}) also provides a very
useful overview of the key issues in this reading course for a lay
audience. J. M. Abowd (\protect\hyperlink{ref-Abowd:JPC:2017}{2017}) and
J. M. Abowd (\protect\hyperlink{ref-abowd:fcsm:2016}{2016}) survey
formal privacy systems being implemented at the Census Bureau.

\hypertarget{formal-privacy}{\section{Formal
Privacy}\label{formal-privacy}}

The literature on formal privacy is vast. Here, we focus on those papers
that will help orient the reader to the key ideas and tools of
differential privacy, particularly those relevant to the economic
problem of determining optimal privacy protection when publishing
population statistics. For the purpose of this reading course and
associated bibliography, we associate formal privacy as a literature
emerging in computer science out of cryptography. In the section on
``\protect\hyperlink{statistical-disclosure-limitation}{Statistical
Disclosure Limitation}'', we recommend additional readings from the SDL
literature, which has a distinct origin in statistics.

Dwork, McSherry, Nissim, \& Smith
(\protect\hyperlink{ref-Dworketal:2006}{2006}) is generally regarded as
the first paper to formally introduce differential privacy. Its
development was due, in part, to the \emph{database reconstruction
theorem} published by Dinur \& Nissim
(\protect\hyperlink{ref-Dinur:2003:RIW:773153.773173}{2003}), which
showed that most data publication mechanisms are blatantly non-private
in a well-defined sense. The database reconstruction theorem has
recently been shown to have very practical consequences for the
statistical disclosure protections in place at the Census Bureau. The
concepts of \emph{k-anonymity} (Sweeney,
\protect\hyperlink{ref-sweeney:2002}{2002}) and \emph{l-diversity}
(Ashwin Machanavajjhala, Kifer, Gehrke, \& Venkitasubramaniam,
\protect\hyperlink{ref-Machanavajjhala:2007:LDP:1217299.1217302}{2007})
are important antecedents that bridge the formal privacy and SDL
literatures.

In assessing formal privacy as a framework for modeling data
publication, it is natural to consider the optimal, or maximal amount of
data accuracy that can be provided while maintaining privacy. Gupta,
Roth, \& Ullman
(\protect\hyperlink{ref-Gupta:2012:ICP:2238936.2238961}{2012}) establish
a mechanism that is universally optimal for a broad class of data users,
suggesting that technical efficiency could be guaranteed in private data
publication without the need to learn about the preferences or side
information of data consumers. However, Brenner \& Nissim
(\protect\hyperlink{ref-BrennerNissim:Impossibility:SIAM:2014}{2014})
showed that their result is not generalizable to broader types of data
publication. Nissim, Orlandi, \& Smorodinsky
(\protect\hyperlink{ref-Nissim:2012:PMD:2229012.2229073}{2012}) provide
a clear and instructive illustration of how individual preferences for
privacy can be modeled in mechanism design problems.

Several papers are more directly relevant to understanding how privacy
affects the work of statistical agencies. Cummings, Echenique, \&
Wierman (\protect\hyperlink{ref-cummings:empirical:corr:2014}{2014})
argue that privacy concerns can affect the way people report data, and
show how, if properly designed, privacy protection may mitigate
misreporting. While there are vast number of papers on the
implementation of differentially private publication algorithms, a few
are particularly relevant to how statistical agencies operate. Hardt,
Ligett, \& McSherry
(\protect\hyperlink{ref-Hardt:Simple:NIPS:2012}{2012}) and Hardt \&
Rothblum (\protect\hyperlink{ref-Hardt:Multiplicative:FOCS10}{2010})
provide the methods and theory for publication through online query
systems. One problem with these methods is that their implementation
depends on the underlying data. By contrast, C. Li, Miklau, Hay,
McGregor, \& Rastogi (\protect\hyperlink{ref-li:matrix:VLDB:2015}{2015})
introduce the Matrix Mechanism, which is data-independent, but also can
provide high accuracy for reasonable levels of privacy. This is one of
the methods under development for use with the 2020 Decennial Census.
Other formal privacy systems currently in use at Census are documented
in A. Machanavajjhala, Kifer, Abowd, Gehrke, \& Vilhuber
(\protect\hyperlink{ref-Machanavajjhala:OTM:ICDE:2008}{2008}) and Haney
et al. (\protect\hyperlink{ref-HaneySIGMOD2017}{2017}).

Finally, so-called ``privacy semantics'' are mappings between
mathematical definitions of privacy and plain language. These are really
important for practitioners because there is usually a gap between how
laypeople think about privacy and how it is defined in the CS
literature. By way of introduction, we recommend He, Machanavajjhala, \&
Ding (\protect\hyperlink{ref-He:Blowfish:ACMSIGMOD:2014}{2014}),
Jorgensen, Yu, \& Cormode
(\protect\hyperlink{ref-jorgensen:personalized:ICDE:2015}{2015}).

\hypertarget{policy-and-official-statistics}{\section{Policy and
Official Statistics}\label{policy-and-official-statistics}}

In January 2017, the Committee on National Statistics released a special
panel report focused on developing innovations in the U.S. statistical
system focused, in part, on preserving privacy (National Academies of
Sciences, Engineering, and Medicine,
\protect\hyperlink{ref-groves:harris-kojetin:2017}{2017}). Their report
is essential reading to understand the governing principles and
practical needs of the statistical system, particularly as it relates to
privacy modernization. For a more applied perspective, Schmutte \&
Vilhuber (\protect\hyperlink{ref-Proc:NSF-Sloan:2017}{2017}) report the
proceedings of an ad hoc workshop on practical privacy held at the
Census Bureau. That workshop gathered together academic privacy
researchers and Census domain experts to help design formal privacy
systems for key data products. In such meetings, it is necessary to make
sure people are speaking the same language. Prewitt
(\protect\hyperlink{ref-prewitt:2011}{2011}) describes the specific
meanings of the terms ``privacy'' and ``confidentiality'' as they have
historically been used at the Census Bureau.

Manski (\protect\hyperlink{ref-Manski2014}{2015}) offers a framework for
thinking about total error in official statistics, which refers to the
various ways measured quantities may differ from the concepts of
interest, including measurement error, design error, and sampling error.
From this perspective, privacy protection is yet another source of error
in any statistical system. Maintaining the public trust is a key factor
motivating the interest of statistical agencies in privacy protection.
The less people trust the system, the less likely they respond
accurately, or at all. Childs, King, \& Fobia
(\protect\hyperlink{ref-Childs:Confidence:SP:2015}{2015}) discuss recent
statistics on trust in official statistics and their implications for
data collection. Finally, Haney et al.
(\protect\hyperlink{ref-HaneySIGMOD2017}{2017}) and Holan, Toth,
Ferreira, \& Karr (\protect\hyperlink{ref-Holan2010}{2010}) are good
examples of the sorts of implementation details one may encounter when
applying statistical privacy protections in public data.

\hypertarget{statistical-disclosure-limitation}{\section{Statistical
Disclosure Limitation}\label{statistical-disclosure-limitation}}

Within most national statistical systems, the primary approach to
protecting respondent privacy has been \emph{statistical disclosure
limitation} or SDL. Anderson \& Seltzer
(\protect\hyperlink{ref-anderson:challenges:JOS:2007}{2007}) describes
the history of threats to confidentiality in the U.S. statistical system
prior to 1965. Fellegi (\protect\hyperlink{ref-Fellegi:1972}{1972})
initiated the statistical analysis of data confidentiality. Dalenius
(\protect\hyperlink{ref-Dalenius:Towards:1977}{1977}) recognized that
statistical agencies would need to do more than just protect against
direct disclosures, and thus warned against what he called inferential
disclosure. His idea was formalized by Duncan \& Lambert
(\protect\hyperlink{ref-Duncan:Lambert:1986}{1986}), and provides the
ultimate rationale for formal privacy in national statistics.

J. M. Abowd \& Schmutte
(\protect\hyperlink{ref-abowdschmutteBPEA2015}{2015}) review the SDL
methods currently in use and discuss their application to economic data.
They argue that the analysis of SDL-laden data is inherently compromised
because the details of the SDL protections cannot be disclosed. If they
cannot be disclosed, their consequences for inference are unknowable,
and, as they show, potentially large. Garfinkel
(\protect\hyperlink{ref-Garfinkel:Deidentification:NIST:2015}{2015})
discusses techniques for de-identifying data and the many ways in which
modern computing tools and a data-rich environment may render effective
de-identification impossible. Finally, Harris-Kojetin et al.
(\protect\hyperlink{ref-spwp22}{2005}) provides the most comprehensive
review of SDL methods currently in use across the U.S. statistical
system.

\hypertarget{economics-of-privacy}{\section{Economics of
Privacy}\label{economics-of-privacy}}

There is a large and robust literature on privacy in economics. That
literature is generally focused on the value to individuals of being
able to conceal private information, like a health condition, from a
firm or prospective employer. The challenge to the firm is to design
mechanisms, like pricing strategies, that encourage people to disclose
private information. For an overview of ideas in this literature, we
recommend reading Stigler
(\protect\hyperlink{ref-stigler1980introduction}{1980}), Posner
(\protect\hyperlink{ref-posner1981economics}{1981}), and Hirshleifer
(\protect\hyperlink{ref-hirshleifer1980privacy}{1980}), which appeared
in an early symposium. Varian (\protect\hyperlink{ref-Varian2002}{2002})
and Acquisti, Taylor, \& Wagman
(\protect\hyperlink{ref-acquisti:taylor:wagman:2015}{2016}) both provide
comprehensive surveys at different points in the development of this
literature.

Very few papers tie the economics of privacy directly to official
statistics. Campbell, Goldfarb, \& Tucker
(\protect\hyperlink{ref-Campbell:Privacy:JEMS}{2015}) discuss the
consequences for firm profits and individual welfare of data privacy
restrictions in California, which prevent firms from sharing certain
types of mortgage data. Goldfarb \& Tucker
(\protect\hyperlink{ref-Goldfarb:Shifts:AERPP:2012}{2012}) discuss
shifts in privacy attitudes and their implications for firms. Hsu et al.
(\protect\hyperlink{ref-Hsu:EconomicEpsilon:IEEE:2014}{2014}), address
the question of optimal privacy protection from a very similar
perspective to J. M. Abowd \& Schmutte
(\protect\hyperlink{ref-AbowdSchmutte:Privacy:AER}{2019}). Finally, Ohm
(\protect\hyperlink{ref-Ohm:Broken:UCLALR:2010}{2010}) surveys the
economic implications of contemporary threats to data privacy from a
legal perspective.

\hypertarget{value-of-privacy-and-data-accuracy}{\section{Value of
Privacy and Data Accuracy}\label{value-of-privacy-and-data-accuracy}}

One key challenge for implementing formal privacy systems lies in
choosing the amount, or type, of privacy to provide. Answering this
question requires some way to understand the individual and social value
of privacy. Ghosh \& Roth
(\protect\hyperlink{ref-Ghosh:Auction:GEB:2015}{2015}) and C. Li, Li,
Miklau, \& Suciu (\protect\hyperlink{ref-Li2014}{2014}) both model
mechanisms for pricing private data under the assumption that
individuals are only willing to disclose such information if they are
paid.

Part of the social value of privacy arises from its relationship to
scientific integrity. While the law of information recovery suggests
that improved privacy must come at the cost of increased error in
published statistics, these effects might be mitigated through two
distinct channels. First, people are more truthful in surveys if they
believe their data is not at risk, as Couper, Singer, Conrad, \& Groves
(\protect\hyperlink{ref-couper2008risk}{2008}) illustrate. Second, work
in computer science (Dwork et al.,
\protect\hyperlink{ref-Dwork:Generalization:NIPS:2015}{2015} ) and
statistics (Cummings, Ligett, Nissim, Roth, \& Wu,
\protect\hyperlink{ref-cummings:adaptive:corr:2016}{2016}) suggests
another somewhat surprising benefit of differential privacy: protection
against overfitting. A complete accounting of the costs and benefits of
formal privacy systems should take these channels into account.

It is equally necessary to develop a more robust understanding of why
data is valuable in the first place, the overall social cost of
increasing error in public statistics. This seems to be an area in which
very little comprehensive theoretical or empirical research has been
done. We nevertheless recommend what seem to be good starting points.

On the theoretical side, economists studying privacy have also developed
models of the value of data to firms. In these models, firms benefit
from being able to tailor prices based on individual demand (C. R.
Taylor, \protect\hyperlink{ref-Taylor:Consumer:RAND:2004}{2004}), or
from being able to market more effectively (Varian,
\protect\hyperlink{ref-Varian1998}{1998}). More recently, a theoretical
literature on information design has begun to consider more effective
ways to manage markets for consumer information, see Bergemann, Bonatti,
\& Smolin (\protect\hyperlink{ref-10.1257ux2faer.20161079}{2018}) and
Pomatto, Strack, \& Tamuz
(\protect\hyperlink{ref-PomattoTheCostOfInformation2018}{2018}). The
recent literature is related to Bruce D. Spencer
(\protect\hyperlink{ref-Spencer:Optimal:JASA:1985}{1985}), who developed
a decision-theoretic framework for modeling optimal data quality.

On the empirical side, a handful of interesting use cases suggest
techniques for uncovering the value of data. For example, Card, Mas,
Moretti, \& Saez (\protect\hyperlink{ref-CardAER2012}{2012}) and
Perez-Truglia
(\protect\hyperlink{ref-perez-truglia:transparency:SSRN:2016}{2016})
show how workers respond to pay transparency policies, which give them
new information about co-worker salaries. Bruce David Spencer \& Seeskin
(\protect\hyperlink{ref-Seeskin:Spencer:Effects:2015}{2015}) use a
calibration exercise to study the costs, measured in misallocated
congressional seats, of reduced accuracy in population census data.

\section{References}\label{references}

\hypertarget{refs}{}
\hypertarget{ref-title13}{}
13 U.S.~Code. (1954). \emph{USC: Title 13 - Census Act}. Retrieved from
\url{https://www.law.cornell.edu/uscode/pdf/lii_usc_TI_13.pdf}

\hypertarget{ref-abowd:fcsm:2016}{}
Abowd, J. M. (2016). Why statistical agencies need to take privacy-loss
budgets seriously, and what it means when they do. \emph{The 13th
Biennial Federal Committee on Statistical Methodology (FCSM) Policy
Conference}. Retrieved from
\url{https://digitalcommons.ilr.cornell.edu/ldi/32/}

\hypertarget{ref-Abowd:JPC:2017}{}
Abowd, J. M. (2017). How will statistical agencies operate when all data
are private? \emph{Journal of Privacy and Confidentiality}, \emph{7}(3).
\url{https://doi.org/10.29012/jpc.v7i3.404}

\hypertarget{ref-abowdschmutteBPEA2015}{}
Abowd, J. M., \& Schmutte, I. M. (2015). Economic analysis and
statistical disclosure limitation. \emph{Brookings Papers on Economic
Activity}, 221--267. \url{https://doi.org/10.1353/eca.2016.0004}

\hypertarget{ref-AbowdSchmutte:Privacy:AER}{}
Abowd, J. M., \& Schmutte, I. M. (2019). An economic analysis of privacy
protection and statistical accuracy as social choices. \emph{American
Economic Review}, \emph{109}(1), 171--202.
\url{https://doi.org/10.1257/aer.20170627}

\hypertarget{ref-acquisti:taylor:wagman:2015}{}
Acquisti, A., Taylor, C., \& Wagman, L. (2016). The economics of
privacy. \emph{Journal of Economic Literature}, \emph{54}(2), 442--492.
\url{https://doi.org/10.1257/jel.54.2.442}

\hypertarget{ref-anderson:challenges:JOS:2007}{}
Anderson, M., \& Seltzer, W. (2007). Challenges to the confidentiality
of US federal statistics, 1910-1965. \emph{Journal of Official
Statistics}, \emph{23}(1), 1. Retrieved from
\url{https://www.scb.se/contentassets/ff271eeeca694f47ae99b942de61df83/challenges-to-the-confidentiality-of-u.s.-federal-statistics-1910-1965.pdf}

\hypertarget{ref-10.1257ux2faer.20161079}{}
Bergemann, D., Bonatti, A., \& Smolin, A. (2018). The design and price
of information. \emph{American Economic Review}, \emph{108}(1), 1--48.
\url{https://doi.org/10.1257/aer.20161079}

\hypertarget{ref-BrennerNissim:Impossibility:SIAM:2014}{}
Brenner, H., \& Nissim, K. (2014). Impossibility of differentially
private universally optimal mechanisms. \emph{SIAM Journal on
Computing}, \emph{43}(5), 1513--1540.
\url{https://doi.org/10.1137/110846671}

\hypertarget{ref-Campbell:Privacy:JEMS}{}
Campbell, J., Goldfarb, A., \& Tucker, C. (2015). Privacy regulation and
market structure. \emph{Journal of Economics \& Management Strategy},
\emph{24}(1), 47--73. \url{https://doi.org/10.1111/jems.12079}

\hypertarget{ref-CardAER2012}{}
Card, D., Mas, A., Moretti, E., \& Saez, E. (2012). Inequality at work:
The effect of peer salaries on job satisfaction. \emph{American Economic
Review}, \emph{102}(6), 2981--3003.
\url{https://doi.org/10.1257/aer.102.6.2981}

\hypertarget{ref-Childs:Confidence:SP:2015}{}
Childs, J. H., King, R., \& Fobia, A. (2015). Confidence in U.S. federal
statistical agencies. \emph{Survey Practice}, \emph{8}(5).
\url{https://doi.org/10.29115/sp-2015-0024}

\hypertarget{ref-couper2008risk}{}
Couper, M. P., Singer, E., Conrad, F. G., \& Groves, R. M. (2008). Risk
of disclosure, perceptions of risk, and concerns about privacy and
confidentiality as factors in survey participation. \emph{Journal of
Official Statistics}, \emph{24}(2), 255. Retrieved from
\url{https://www.ncbi.nlm.nih.gov/pmc/articles/PMC3096944/}

\hypertarget{ref-cummings:empirical:corr:2014}{}
Cummings, R., Echenique, F., \& Wierman, A. (2014). The empirical
implications of privacy-aware choice. \emph{CoRR}, \emph{abs/1401.0336}.
Retrieved from \url{http://arxiv.org/abs/1401.0336}

\hypertarget{ref-cummings:adaptive:corr:2016}{}
Cummings, R., Ligett, K., Nissim, K., Roth, A., \& Wu, Z. S. (2016).
Adaptive learning with robust generalization guarantees. \emph{CoRR},
\emph{abs/1602.07726}. Retrieved from
\url{http://arxiv.org/abs/1602.07726}

\hypertarget{ref-Dalenius:Towards:1977}{}
Dalenius, T. (1977). Towards a methodology for statistical disclosure
control. \emph{Statistik Tidskrift}, \emph{15}, 429--444.
\url{https://doi.org/10.1145/320613.320616}

\hypertarget{ref-Dinur:2003:RIW:773153.773173}{}
Dinur, I., \& Nissim, K. (2003). Revealing information while preserving
privacy. \emph{Proceedings of the Twenty-second ACM SIGMOD-SIGACT-SIGART
Symposium on Principles of Database Systems}, 202--210.
\url{https://doi.org/10.1145/773153.773173}

\hypertarget{ref-Duncan:Lambert:1986}{}
Duncan, G., \& Lambert, D. (1986). Disclosure-limited data
dissemination. \emph{Journal of the American Statistical Association},
\emph{81}(393), 10--18.
\url{https://doi.org/10.1080/01621459.1986.10478229}

\hypertarget{ref-Dwork:Roth:journal:version:2014}{}
Dwork, C., \& Roth, A. (2014). The Algorithmic Foundations of
Differential Privacy. \emph{Foundations and Trends in Theoretical
Computer Science}, \emph{9}(3-4), 211--407.
\url{https://doi.org/10.1561/0400000042}

\hypertarget{ref-Dwork:Generalization:NIPS:2015}{}
Dwork, C., Feldman, V., Hardt, M., Pitassi, T., Reingold, O., \& Roth,
A. (2015). Generalization in adaptive data analysis and holdout reuse.
In C. Cortes, N. D. Lawrence, D. D. Lee, M. Sugiyama, \& R. Garnett
(Eds.), \emph{Advances in neural information processing systems 28} (pp.
2341--2349). Retrieved from
\url{http://papers.nips.cc/paper/5993-generalization-in-adaptive-data-analysis-and-holdout-reuse.pdf}

\hypertarget{ref-Dworketal:2006}{}
Dwork, C., McSherry, F., Nissim, K., \& Smith, A. (2006). Calibrating
Noise to Sensitivity in Private Data Analysis. \emph{Proceedings of the
Third conference on Theory of Cryptography}, 265--284.
\url{https://doi.org/10.1007/11681878_14}

\hypertarget{ref-Fellegi:1972}{}
Fellegi, I. P. (1972). On the question of statistical confidentiality.
\emph{Journal of the American Statistical Association}, \emph{67}(337),
7--18. \url{https://doi.org/10.2307/2284695}

\hypertarget{ref-Garfinkel:Deidentification:NIST:2015}{}
Garfinkel, S. (2015). \emph{De-Identification of personal information}
(Internal Report No. 8053). \url{https://doi.org/10.6028/nist.ir.8053}

\hypertarget{ref-Ghosh:Auction:GEB:2015}{}
Ghosh, A., \& Roth, A. (2015). Selling privacy at auction. \emph{Games
and Economic Behavior}, \emph{91}, 334--346.
\url{https://doi.org/10.1016/j.geb.2013.06.013}

\hypertarget{ref-Goldfarb:Shifts:AERPP:2012}{}
Goldfarb, A., \& Tucker, C. (2012). Shifts in privacy concerns.
\emph{American Economic Review}, \emph{102}(3), 349--353.
\url{https://doi.org/10.1257/aer.102.3.349}

\hypertarget{ref-Goroff2015}{}
Goroff, D. L. (2015). Balancing privacy versus accuracy in research
protocols. \emph{Science}, \emph{347}(6221), 479--480.
\url{https://doi.org/10.1126/science.aaa3483}

\hypertarget{ref-Gupta:2012:ICP:2238936.2238961}{}
Gupta, A., Roth, A., \& Ullman, J. (2012). Iterative constructions and
private data release. \emph{Proceedings of the 9th International
Conference on Theory of Cryptography}, 339--356.
\url{https://doi.org/10.1007/978-3-642-28914-9_19}

\hypertarget{ref-cipsea}{}
H.R.4174. (2018). \emph{Confidential Information Protection and
Statistical Efficency Act}. Retrieved from
\url{https://www.congress.gov/bill/115th-congress/house-bill/4174}

\hypertarget{ref-HaneySIGMOD2017}{}
Haney, S., Machanavajjhala, A., Abowd, J. M., Graham, M., Kutzbach, M.,
\& Vilhuber, L. (2017). Utility cost of formal privacy for releasing
national employer-employee statistics. In \emph{SIGMOD '17}.
\emph{Proceedings of the 2017 International Conference on Management of
Data}. \url{https://doi.org/10.1145/3035918.3035940}

\hypertarget{ref-Hardt:Multiplicative:FOCS10}{}
Hardt, M., \& Rothblum, G. N. (2010). A multiplicative weights mechanism
for privacy-preserving data analysis. \emph{2010 IEEE 51st Annual
Symposium on Foundations of Computer Science}, 61--70.
\url{https://doi.org/10.1109/FOCS.2010.85}

\hypertarget{ref-Hardt:Simple:NIPS:2012}{}
Hardt, M., Ligett, K., \& McSherry, F. (2012). A Simple and Practical
Algorithm for Differentially Private Data Release. In F. Pereira, C.
Burges, L. Bottou, \& K. Weinberger (Eds.), \emph{Advances in neural
information processing systems 25} (pp. 2339--2347). Retrieved from
\url{http://papers.nips.cc/paper/4548-a-simple-and-practical-algorithm-for-differentially-private-data-release.pdf}

\hypertarget{ref-spwp22}{}
Harris-Kojetin, B. A., Alvey, W. L., Carlson, L., Cohen, S. B., Cohen,
S. H., Cox, L. H., \ldots{} Groves, R. (2005). \emph{Statistical policy
working paper 22: Report on statistical disclosure limitation
methodology} {[}Research Report{]}. Retrieved from U.S. Federal
Committee on Statistical Methodology website:
\url{https://nces.ed.gov/FCSM/pdf/spwp22.pdf}

\hypertarget{ref-Harvard:PrivacyTools}{}
Harvard University Privacy Tools Project. (2019). \emph{Homepage}.
Retrieved from \url{https://privacytools.seas.harvard.edu/}

\hypertarget{ref-He:Blowfish:ACMSIGMOD:2014}{}
He, X., Machanavajjhala, A., \& Ding, B. (2014). Blowfish privacy:
Tuning privacy-utility trade-offs using policies. In \emph{Proceedings
of the acm sigmod international conference on management of data} (pp.
1447--1458). \url{https://doi.org/10.1145/2588555.2588581}

\hypertarget{ref-Heffetz2014}{}
Heffetz, O., \& Ligett, K. (2014). Privacy and data-based research.
\emph{Journal of Economic Perspectives}, \emph{28}(2), 75--98.
\url{https://doi.org/10.1257/jep.28.2.75}

\hypertarget{ref-hirshleifer1980privacy}{}
Hirshleifer, J. (1980). Privacy: Its origin, function, and future.
\emph{The Journal of Legal Studies}, 649--664.
\url{https://doi.org/10.1086/467659}

\hypertarget{ref-Holan2010}{}
Holan, S. H., Toth, D., Ferreira, M. A. R., \& Karr, A. F. (2010).
Bayesian multiscale multiple imputation with implications for data
confidentiality. \emph{Journal of the American Statistical Association},
\emph{105}(490), 564--577.
\url{https://doi.org/10.1198/jasa.2009.ap08629}

\hypertarget{ref-Hsu:EconomicEpsilon:IEEE:2014}{}
Hsu, J., Gaboardi, M., Haeberlen, A., Khanna, S., Narayan, A., Pierce,
B. C., \& Roth, A. (2014). Differential privacy: An economic method for
choosing epsilon. \emph{2014 IEEE 27th Computer Security Foundations
Symposium}, 398--410. \url{https://doi.org/10.1109/CSF.2014.35}

\hypertarget{ref-jones:2017}{}
Jones, C. (2017). \emph{Nonconfidential memorandum on Census Bureau
privacy breaches}. Retrieved from
\url{http://doi.org/10.5281/zenodo.1345775}

\hypertarget{ref-jorgensen:personalized:ICDE:2015}{}
Jorgensen, Z., Yu, T., \& Cormode, G. (2015). Conservative or liberal?
Personalized differential privacy. \emph{2015 IEEE 31st International
Conference on Data Engineering}, 1023--1034.
\url{https://doi.org/10.1109/ICDE.2015.7113353}

\hypertarget{ref-Li2014}{}
Li, C., Li, D. Y., Miklau, G., \& Suciu, D. A. N. (2014). A theory of
pricing private data. \emph{ACM Transactions on Database Systems},
\emph{39}(4), 34:1--34:27. \url{https://doi.org/10.1145/2448496.2448502}

\hypertarget{ref-li:matrix:VLDB:2015}{}
Li, C., Miklau, G., Hay, M., McGregor, A., \& Rastogi, V. (2015). The
matrix mechanism: Optimizing linear counting queries under differential
privacy. \emph{The VLDB Journal}, \emph{24}(6), 757--781.
\url{https://doi.org/10.1007/s00778-015-0398-x}

\hypertarget{ref-Machanavajjhala:OTM:ICDE:2008}{}
Machanavajjhala, A., Kifer, D., Abowd, J., Gehrke, J., \& Vilhuber, L.
(2008). Privacy: Theory meets practice on the map. \emph{Proceedings of
the 2008 ieee 24th international conference on data engineering},
277--286. \url{https://doi.org/10.1109/ICDE.2008.4497436}

\hypertarget{ref-Machanavajjhala:2007:LDP:1217299.1217302}{}
Machanavajjhala, A., Kifer, D., Gehrke, J., \& Venkitasubramaniam, M.
(2007). L-diversity: Privacy beyond k-anonymity. \emph{ACM Transactions
on Knowledge Discovery from Data}, \emph{1}(1).
\url{https://doi.org/10.1145/1217299.1217302}

\hypertarget{ref-Manski2014}{}
Manski, C. F. (2015). Communicating uncertainty in official economic
statistics: An appraisal fifty years after morgenstern. \emph{Journal of
Economic Literature}, \emph{53}(3), 631--653.
\url{https://doi.org/10.1257/jel.53.3.631}

\hypertarget{ref-groves:harris-kojetin:2017}{}
National Academies of Sciences, Engineering, and Medicine. (2017).
\emph{Innovations in federal statistics: Combining data sources while
protecting privacy}. \url{https://doi.org/10.17226/24652}

\hypertarget{ref-Nissim:2012:PMD:2229012.2229073}{}
Nissim, K., Orlandi, C., \& Smorodinsky, R. (2012). Privacy-aware
mechanism design. \emph{Proceedings of the 13th acm conference on
electronic commerce}, 774--789.
\url{https://doi.org/10.1145/2229012.2229073}

\hypertarget{ref-Ohm:Broken:UCLALR:2010}{}
Ohm, P. (2010). Broken promises of privacy: Responding to the surprising
failure of anonymization. \emph{UCLA Law Review}, \emph{57}, 1701.

\hypertarget{ref-perez-truglia:transparency:SSRN:2016}{}
Perez-Truglia, R. (2016). The effects of income transparency on
well-being: Evidence from a natural experiment. \emph{SSRN}.
\url{https://doi.org/10.2139/ssrn.2657808}

\hypertarget{ref-PomattoTheCostOfInformation2018}{}
Pomatto, L., Strack, P., \& Tamuz, O. (2018). \emph{The cost of
information}. arXiv.

\hypertarget{ref-posner1981economics}{}
Posner, R. A. (1981). The economics of privacy. \emph{The American
Economic Review}, 405--409.

\hypertarget{ref-prewitt:2011}{}
Prewitt, K. (2011). Why It Matters to Distinguish Between Privacy \&
Confidentiality. \emph{Journal of Privacy and Confidentiality},
\emph{3}(2), 41--47. \url{https://doi.org/10.29012/jpc.v3i2.600}

\hypertarget{ref-Proc:NSF-Sloan:2017}{}
Schmutte, I. M., \& Vilhuber, L. (Eds.). (2017). \emph{Proceedings from
the 2016 NSF-Sloan Workshop on Practical Privacy}. Retrieved from
\url{https://digitalcommons.ilr.cornell.edu/ldi/33/}

\hypertarget{ref-Spencer:Optimal:JASA:1985}{}
Spencer, B. D. (1985). Optimal data quality. \emph{Journal of the
American Statistical Association}, \emph{80}(391), 564--573.
\url{https://doi.org/10.1080/01621459.1985.10478155}

\hypertarget{ref-Seeskin:Spencer:Effects:2015}{}
Spencer, B. D., \& Seeskin, Z. H. (2015). Effects of Census accuracy on
apportionment of Congress and allocations of federal funds. \emph{JSM
Proceedings, Government Statistics Section}, 3061--3075. Retrieved from
\url{https://www.ipr.northwestern.edu/publications/papers/2015/ipr-wp-15-05.html}

\hypertarget{ref-stigler1980introduction}{}
Stigler, G. J. (1980). An introduction to privacy in economics and
politics. \emph{Journal of Legal Studies}, \emph{9}(4), 623--644.
\url{https://doi.org/10.2307/724174}

\hypertarget{ref-sweeney:2002}{}
Sweeney, L. (2002). Achieving k-anonymity privacy protection using
generalization and suppression. \emph{International Journal on
Uncertainty, Fuzziness and Knowledge-Based Systems}, \emph{10}(5),
571--588. \url{https://doi.org/10.1142/s021848850200165x}

\hypertarget{ref-Taylor:Consumer:RAND:2004}{}
Taylor, C. R. (2004). Consumer privacy and the market for customer
information. \emph{The RAND Journal of Economics}, \emph{35}(4),
631--650. \url{https://doi.org/10.2307/1593765}

\hypertarget{ref-Varian1998}{}
Varian, H. R. (1998). \emph{Markets for Information Goods} (pp. 1--19)
{[}Mimeo{]}. Retrieved from UC Berkeley School of Information website:
\url{http://people.ischool.berkeley.edu/~hal/Papers/japan/index.html}

\hypertarget{ref-Varian2002}{}
Varian, H. R. (2002). Economic aspects of personal privacy. In W. H.
Lehr \& L. M. Pupillo (Eds.), \emph{Cyber policy and economics in an
internet age} (pp. 127--137).
\url{https://doi.org/10.1007/978-1-4757-3575-8_9}

\hypertarget{ref-WoodVanderbiltJ.Entertain.Technol.Law2018}{}
Wood, A., Altman, M., Bembenek, A., Bun, M., Gaboardi, M., Honaker, J.,
\ldots{} Vadhan, S. (2018). Differential Privacy: A Primer for a
Non-Technical Audience. \emph{Vanderbilt Journal of Entertainment and
Technology Law}, \emph{21}(1). Retrieved from
\url{http://www.jetlaw.org/journal-archives/volume-21/volume-21-issue-1/differential-privacy-a-primer-for-a-non-technical-audience/}


\end{document}
